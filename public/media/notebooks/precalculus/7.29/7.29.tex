\documentclass[12pt]{article}

\usepackage[utf8]{inputenc}
\usepackage{latexsym,amsfonts,amssymb,amsthm,amsmath}
\usepackage{float}

\setlength{\parindent}{0in}
\setlength{\oddsidemargin}{0in}
\setlength{\textwidth}{6.5in}
\setlength{\textheight}{8.8in}
\setlength{\topmargin}{0in}
\setlength{\headheight}{18pt}
\usepackage{graphicx}
\usepackage{tikz}

\usepackage{hyperref}
\hypersetup{
    colorlinks=true,
    linkcolor=blue,
    filecolor=magenta,      
    urlcolor=cyan,
    pdftitle={Overleaf Example},
    pdfpagemode=FullScreen,
}

\urlstyle{same}

\usepackage{caption}
\DeclareCaptionFormat{citation}{%
  \ifx\captioncitation\relax\relax\else
    \captioncitation\par
  \fi
  #1#2#3\par}
\newcommand*\setcaptioncitation[1]{\def\captioncitation{\textit{Source:}~#1}}
\let\captioncitation\relax
\captionsetup{format=citation,justification=centering}

\title{MATH1034OL1 Pre-Calculus Mathematics Notes from Sections 5.4 (Monday)}
\author{Elijah Renner}

\begin{document}

\maketitle

\vspace{0.5in}

\tableofcontents

\section{Exponential Growth and Decay with the Natural Base}

Today's class only introduced the idea of exponential growth and decay, so I'll cover that here.\\

We define the general equation for exponential growth as\\

\[y(t)=ae^{kt}\]\\

where \(y(t)\) is the value at time \(t\), \(a\) is the initial value, and \(k\) is the growth constant.\\

Usually, we'll need to solve for \(k\) given information.\\

Problem: when will a population reach 50000 people if there are 10000 people initially and triples every two years?\\

Since the questions gives us the growth rate ("triples every two years"), we'll create the datapoint (2,30000) which repreesnts the population after two years.\\ 

We'll then plug this and given information into our equation \(y(t)=ae^{kt}\) and solve for \(k\), the growth constant:\\

\[30000=10000e^{2k}\]
\[\implies 3=e^{2k}\]
\[\implies \ln 3=\ln e^{2k}=2k\]
\[\implies k=\frac{\ln 3}{2}\]\\

Now we can write the population as an exponential function of time:\\

\[y(t)=10000e^{kt}=10000e^{\frac{\ln 3}{2}t}\]\\

The question asks for the time when the population \(y(t)\) is 50000, so we plug that in:\\

\[50000=10000e^{\frac{\ln 3}{2}t}\]
\[\implies 5=e^{\frac{\ln 3}{2}t}\]
\[\implies \ln 5=\ln e^{\frac{\ln 3}{2}t}=\frac{\ln 3}{2}t}\]
\[\implies t=\frac{\ln 5}{\left(\frac{\ln 3}{2}\right)}=\frac{2\ln 5}{\ln 3}}\text{ years}\]\\

Also, \(e\) is the base since it's easily differentiable in calculus. I had to look this up because I was confused why we didn't use simple bases (like \(\frac{1}{2}\)) for half-life decay.\\

Have a great week! 

\end{document}