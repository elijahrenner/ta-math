\documentclass[12pt]{article}

\usepackage[utf8]{inputenc}
\usepackage{latexsym,amsfonts,amssymb,amsthm,amsmath}
\usepackage{float}

\setlength{\parindent}{0in}
\setlength{\oddsidemargin}{0in}
\setlength{\textwidth}{6.5in}
\setlength{\textheight}{8.8in}
\setlength{\topmargin}{0in}
\setlength{\headheight}{18pt}
\usepackage{graphicx}

\usepackage{caption}
\DeclareCaptionFormat{citation}{%
  \ifx\captioncitation\relax\relax\else
    \captioncitation\par
  \fi
  #1#2#3\par}
\newcommand*\setcaptioncitation[1]{\def\captioncitation{\textit{Source:}~#1}}
\let\captioncitation\relax
\captionsetup{format=citation,justification=centering}


\title{MATH1034OL1 Pre-Calculus Mathematics Notes from Sections 4.6, 2.4 (Wednesday)}
\author{Elijah Renner}

\begin{document}

\maketitle

\vspace{0.5in}

\tableofcontents

\section{Determining Quadrants of Angles by Trigonometric Function Signs}

For these problems, we approach them using our All Students Take Calculus (ASTC) rule. ASTC tells which trig functions are + in which quadrants. Refer to July 5th's notes for a description of it.\\

Problem: given \(\sec\theta>0\) and \(\cot\theta<0\), which quadrant does \(\theta\) lie in?\\

We know that \(\sec\) is the reciprocal of \(\cos\), meaning \(\theta\) can only be in quadrants one or four since it's given that  \(\sec\theta>0\) and \(\cos\) is positive in those quadrants by ASTC.\\

We also know that \(\cot\), the reciprocal of \(\tan\), is only negative in quadrants two and four by ASTC.\\

Only quadrant four satisfies both requirements. Hence, \(\theta\) is in quadrant four.\\

\section{Rationalizing Irrational Numerators and Denominators}

To rationalize the denominator of the expression
\[
\frac{3}{2 + \sqrt{5}},
\]
follow these steps:\\

1. Identify the conjugate of the denominator. The denominator is \(2 + \sqrt{5}\). The conjugate is \(2 - \sqrt{5}\).\\

2. Multiply the numerator and denominator by the conjugate:
\[
\frac{3}{2 + \sqrt{5}} \cdot \frac{2 - \sqrt{5}}{2 - \sqrt{5}}
\]

3. Simplify the expression. First, calculate the product in the denominator:
\[
(2 + \sqrt{5})(2 - \sqrt{5}) = 2^2 - (\sqrt{5})^2 = 4 - 5 = -1
\]
Thus, the expression becomes:
\[
\frac{3(2 - \sqrt{5})}{-1}
\]
Simplify:
\[
\frac{3(2 - \sqrt{5})}{-1} = -3(2 - \sqrt{5}) = -6 + 3\sqrt{5}
\]

\section{Function Operations}

A key piece of notation: \((f \circ g)(x)=f(g(x))\). where \(f(g(x))\) represents inputting \(g\) as \(x\) in \(f(x)\).\\ 

1. \textbf{Sum of Functions} \( (f + g)(x) \)
   - \textbf{Domain}: The domain of \( (f + g)(x) \) is the intersection of the domains of \( f(x) \) and \( g(x) \):
     \[
     \text{Domain}(f + g) = \text{Domain}(f) \cap \text{Domain}(g)
     \]

2. \textbf{Difference of Functions} \( (f - g)(x) \)
   - \textbf{Domain}: The domain of \( (f - g)(x) \) is the intersection of the domains of \( f(x) \) and \( g(x) \):
     \[
     \text{Domain}(f - g) = \text{Domain}(f) \cap \text{Domain}(g)
     \]

3. \textbf{Product of Functions} \( (f \cdot g)(x) \)
   - \textbf{Domain}: The domain of \( (f \cdot g)(x) \) is the intersection of the domains of \( f(x) \) and \( g(x) \):
     \[
     \text{Domain}(f \cdot g) = \text{Domain}(f) \cap \text{Domain}(g)
     \]

4. \textbf{Quotient of Functions} \( \left( \frac{f}{g} \right)(x) \)
   - \textbf{Domain}: The domain of \( \left( \frac{f}{g} \right)(x) \) is the intersection of the domains of \( f(x) \) and \( g(x) \), excluding the points where \( g(x) = 0 \):
     \[
     \text{Domain}\left( \frac{f}{g} \right) = \left( \text{Domain}(f) \cap \text{Domain}(g) \right) \setminus \{ x \mid g(x) = 0 \}
     \]

5. \textbf{Composition of Functions} \( (f \circ g)(x) \)
   - \textbf{Domain}: The domain of \( (f \circ g)(x) \) is the set of all \( x \) in the domain of \( g(x) \) such that \( g(x) \) is in the domain of \( f(x) \):
     \[
     \text{Domain}(f \circ g) = \{ x \in \text{Domain}(g) \mid g(x) \in \text{Domain}(f) \}
     \]
     
\section{Piecewise Functions}

Define the piecewise function \( f(x) \) as follows:
\[
f(x) =
\begin{cases}
x^2 & \text{if } x < 0 \\
2x + 1 & \text{if } 0 \leq x < 3 \\
5 & \text{if } x \geq 3
\end{cases}
\]



The \(x^{2}\), \(2x+1\), and \(5\) represent the value of \(f\) if the condition at right is met. For example, if \(x<0\), \(f(x)=x^{2}\).\\



\end{document}
